
We take the inspiration for our performance-evaluation approach from
 Takikawa et al.~\cite{asumu:practical} present a formative evaluation of
 their gradual typing system for object-oriented Racket. 

 ---------------

describe the lattice in detail, abstractly and concretely, using one of our
example; make sure to discuss the ``completely typed'' issue/problem 

---------------

explain the questions we ask about the lattice: 

{\bf Question 1} how does the performance of the ``completely typed'' configuration
compare to the original, untyped configuration 

{\bf Question 2/a} how many of the lattice points experience no worse than an N\%
slowdown compared to the untyped configuration? We call those N-{\it
delivery ready\/}.

{\bf Question 2/b} how many of the lattice points experience a slow-down of more
than N\% but less than M\% of the untyped configuration? We call those
M-{\it evolutionary acceptable\/}.

{\bf Note 2/c} all other configurations are considered {\it unacceptable\/}

{\bf Question 3} how many of the unacceptable configuration are within L
steps of delivery-ready or evolutionary-acceptable configuration? We call
those L-step acceptable. 


