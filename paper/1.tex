
Over the past couple of decades dynamically-typed languages have become a
 staple of the software engineering world. Programmers use these languages
 to build all kinds of software systems. In many cases, the systems start
 as innocent prototypes. Soon enough, though, they grow into complex,
 multi-module programs, at which point the engineers realize that they are
 facing a maintenance nightmare. The lack of type information makes it
 expensive to reconstruct what the creators had in mind and to maintain the
 implicit invariants of the code. 

Gradual typing~\cite{st:gradual06,thf:dls06} proposes a language-oriented
 solution to this pressing software engineering problem. The basic idea is
 to extend the language so that programmers can incrementally equip
 programs with types. In contrast to languages with optional type systems,
 gradual typing projects insist on type soundness, that is, the type
 information should be meaningful and predict run-time behavior. 

Realizing type soundness in this world requires run-time checks that
 mediate impedance mismatches between the typed and untyped portions of the
 programs. In recognition of the cost of these checks, macro-level gradual
 typing forces programmers to annotate entire modules with types;
 behavioral contracts~\cite{ff:ho-contracts} between typed and untyped
 modules enforce type soundness. Micro-level gradual typing takes the
 approach that an untyped program is typed with all types implicitly
 equated with type {\tt Dyn}. When programmers add fine-grained type
 annotations, the compiler inserts casts that coerce values from {\tt Dyn}
 to these subtypes.

Both approaches to gradual typing come with two implicit claims. First, the
 type systems accommodate the programming idioms that evolved in the
 untyped world. This accommodation allows programmers to add types without
 touching the existing code. Second, the cost of soundness is tolerable,
 meaning programs remain performant even as programmers add type
 annotations. While almost every publication on practical gradual typing
 validates some version of the first claim, no gradual typing project has
 tackled the second claim. Most publications have subtle remarks about the
 performance of partially typed programs; some plainly admit that such
 mixed programs may suffer performance degradations of two orders of
 magnitude. 

This paper introduces a framework for the systematic performance evaluation
 of gradual typing systems. In the context of macro-level gradual typing,
 the basic idea is to simulate the software engineering process on
 multi-module programs. All $n$ modules are annotated with types, and the
 resulting collection of $2 \cdot n$ modules is then used to create all
 $2^n$ configurations. The collection of these configurations forms a
 complete lattice with the completely untyped one at the bottom and the
 completely typed one at the top. In between, the lattice contains
 configurations where some modules are typed and others are untyped. Adding
 types to an untyped module in one of these configurations yields a
 configuration of the next level in the lattice. In short, the lattice
 mimics how a programmer picks one module of many and adds types when a
 maintenance task comes up, how another programmer modifies the next
 module, and so on.

A performance measurement of a gradual typing system must run every
 configuration for every benchmark and determine its average running time.
 The performance evaluation inspects the results of this lattice to answer
 some basic questions.  For the gradual typing system as a whole, we must
 consider how it performs on all of the lattice benchmarks. We use our
 framework to evaluate two implementations of gradual typing on a dozen
 benchmark lattices: the original Typed Racket implementation and the new
 Pycket-based version.

Section~\ref{sec:fwk} explains the performance framework in detail,
 including the information we retrieve from the lattices and how we
 parametrize these retrievals. Next, section~\ref{sec:bm} presents our
 specific Typed Racket benchmarks. Sections~\ref{sec:tr} and~\ref{sec:trp}
 present the numeric results of evaluating Typed Racket's original
 implementation and the Pycket variant,
 respectively. Section~\ref{sec:death} discusses these
 results. Section~\ref{sec:rel} reviews the literature on gradual typing
 systems with respect to performance evaluation. Section~\ref{sec:fut}
 concludes with ideas for future work.
